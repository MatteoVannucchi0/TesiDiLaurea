% !TeX root = main.tex
%! Author = Matteo
%! Date = 10/06/2022

% Preamble
%\documentclass[11pt]{book}

\documentclass[12pt, letterpaper, twoside, openright, a4paper]{book}
% Packages
\usepackage{amsmath}
\usepackage[italian]{babel}
\usepackage{newlfont}
\usepackage{gensymb}
\usepackage{lipsum}

\newcommand\abstractname{Abstract}  %%% here

%Comando per scrivere c++
\newcommand{\CC}{C\nolinebreak\hspace{-.05em}\raisebox{.4ex}{\tiny\bf +}\nolinebreak\hspace{-.10em}\raisebox{.4ex}{\tiny\bf +}}
\def\CC{{C\nolinebreak[4]\hspace{-.05em}\raisebox{.4ex}{\tiny\bf ++}}}

%Comando per scrivere Machine Learning e Deep Learning
\newcommand{\ML}{\textit{Machine Learning}}
\newcommand{\DL}{\textit{Deep Learning}}

% Per il frontespizio
\textwidth=450pt\oddsidemargin=0pt

% Documents
\begin{document}
    \frontmatter
    \begin{titlepage}
\begin{center}
{{\Large{\textsc{Alma Mater Studiorum $\cdot$ Universit\`a di
Bologna}}}} \rule[0.1cm]{15.8cm}{0.1mm}
\rule[0.5cm]{15.8cm}{0.6mm}
{\small{\bf SCUOLA DI SCIENZE\\
Corso di Laurea Triennale in Informatica }}
\end{center}
\vspace{15mm}
\begin{center}
{\LARGE{\bf Tecniche di deep learning}}\\
\vspace{3mm}
{\LARGE{\bf per il riconoscimento}}\\
\vspace{3mm}
{\LARGE{\bf di errori nel codice}}\\
\end{center}
\vspace{40mm}
\par
\noindent
\begin{minipage}[t]{0.47\textwidth}
{\large{\bf Relatore:\\
Chiar.mo Prof.\\
Maurizio Gabbrielli}}
\end{minipage}
\hfill
\begin{minipage}[t]{0.47\textwidth}\raggedleft
{\large{\bf Presentata da:\\
Matteo Vannucchi}}
\end{minipage}
\vspace{20mm}
\begin{center}
{\large{\bf Sessione I\\%inserire il numero della sessione in cui ci si laurea
Anno Accademico 2021-2022}}%inserire l'anno accademico a cui si è iscritti
\end{center}
\end{titlepage}

    \makeatletter
\if@titlepage
  \newenvironment{abstract}{%
      \titlepage
      \null\vfil
      \@beginparpenalty\@lowpenalty
      \begin{center}%
        \bfseries Abstract
        \@endparpenalty\@M
      \end{center}}%
     {\par\vfil\null\endtitlepage}
\else
  \newenvironment{abstract}{%
      \if@twocolumn
        \section*{\abstractname}%
      \else
        \small
        \begin{center}%
          {\bfseries \abstractname\vspace{-.5em}\vspace{\z@}}%
        \end{center}%
        \quotation
      \fi}
      {\if@twocolumn\else\endquotation\fi}
\fi
\makeatother

\begin{abstract}
  Il ruolo dell'informatica, in un mondo in progressiva digitalizzazione di ogni singolo aspetto della vita dell'individuo, è ormai diventato chiave del suo funzionamento. 
  Con l'aumentare della complessità del codice e delle dimensioni dei progetti, il rilevamento di errori diventa sempre di più un'attività difficile e lunga. Meccanismi di analisi
  del codice sorgente tradizionali sono esistiti fin dalla nascita dell'informatica stessa, e il loro ruolo all'interno della catena produttiva di un team di programmatori non è mai stato cosi fondamentale come 
  lo è tuttora. Questi meccanismi di analisi però non sono esenti da problematiche: il tempo di esecuzione su progetti grandi e la percentuale di falsi positivi possono infatti diventare un grosso problema.
  Per questi motivi meccanismi fondati su \ML, e più in particolare \DL, sono stati sviluppati negli ultimi anni. Questo lavoro di tesi si pone quindi l'obbiettivo di esplorare e sviluppare un modello per il riconoscimento di errori
  in un qualsiasi file sorgente scritto in linguaggio sia C sia \CC.


\end{abstract}

    \tableofcontents
    \listoffigures
    \listoftables

    \mainmatter
    \chapter*{Introduzione}
    \chapter{Introduzione teorica}\label{chap:introduzione_teorica}

\section{Code2Vec}\label{sec:code2vec}

\subsection{Meccanismo di attenzione}

\subsection{AstContext}\label{subsec:astcontext}
    \cite{Dirac19}

    \cleardoublepage
    \bibliographystyle{plain}
    \addcontentsline{toc}{chapter}{\bibname}
    \bibliography{main}
\end{document}
