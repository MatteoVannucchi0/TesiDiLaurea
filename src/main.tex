% !TeX root = main.tex
%! Author = Matteo
%! Date = 10/06/2022

% Preamble
%\documentclass[11pt]{book}

\documentclass[12pt, letterpaper, twoside, openright, a4paper]{book}
% Packages
\usepackage{amsmath}
\usepackage[italian]{babel}
\usepackage{newlfont}
\usepackage{gensymb}
\usepackage{lipsum}

\newcommand\abstractname{Abstract}  %%% here

%Comando per scrivere c++
\newcommand{\CC}{C\nolinebreak\hspace{-.05em}\raisebox{.4ex}{\tiny\bf +}\nolinebreak\hspace{-.10em}\raisebox{.4ex}{\tiny\bf +}}
\def\CC{{C\nolinebreak[4]\hspace{-.05em}\raisebox{.4ex}{\tiny\bf ++}}}

%Comando per scrivere Machine Learning e Deep Learning
\newcommand{\ML}{\textit{Machine Learning}}
\newcommand{\DL}{\textit{Deep Learning}}

% Per il frontespizio
\textwidth=450pt\oddsidemargin=0pt

% Documents
\begin{document}
    \frontmatter
    \begin{titlepage}
\begin{center}
{{\Large{\textsc{Alma Mater Studiorum $\cdot$ Universit\`a di
Bologna}}}} \rule[0.1cm]{15.8cm}{0.1mm}
\rule[0.5cm]{15.8cm}{0.6mm}
{\small{\bf SCUOLA DI SCIENZE\\
Corso di Laurea Triennale in Informatica }}
\end{center}
\vspace{15mm}
\begin{center}
{\LARGE{\bf Tecniche di deep learning}}\\
\vspace{3mm}
{\LARGE{\bf per il riconoscimento}}\\
\vspace{3mm}
{\LARGE{\bf di errori nei programmi}}\\
\end{center}
\vspace{40mm}
\par
\noindent
\begin{minipage}[t]{0.47\textwidth}
{\large{\bf Relatore:\\
Chiar.mo Prof.\\
Maurizio Gabbrielli}}
\end{minipage}
\hfill
\begin{minipage}[t]{0.47\textwidth}\raggedleft
{\large{\bf Presentata da:\\
Matteo Vannucchi}}
\end{minipage}
\vspace{20mm}
\begin{center}
{\large{\bf Sessione I\\%inserire il numero della sessione in cui ci si laurea
Anno Accademico 2021-2022}}%inserire l'anno accademico a cui si è iscritti
\end{center}
\end{titlepage}

    \begingroup\selectlanguage{english}
\begin{abstract}
  Il ruolo dell'informatica, in un mondo in progressiva digitalizzazione di ogni singolo aspetto della vita dell'individuo, è ormai diventato chiave del suo funzionamento. 
  Con l'aumentare della complessità del codice e delle dimensioni dei progetti, il rilevamento di errori diventa sempre di più un'attività difficile e lunga. Meccanismi di analisi
  del codice sorgente tradizionali sono esistiti fin dalla nascita dell'informatica stessa, e il loro ruolo all'interno della catena produttiva di un team di programmatori non è mai stato cosi fondamentale come 
  lo è tuttora. Questi meccanismi di analisi però non sono esenti da problematiche: il tempo di esecuzione su progetti grandi e la percentuale di falsi positivi possono infatti diventare un grosso problema.
  Per questi motivi meccanismi fondati su \ML, e più in particolare \DL, sono stati sviluppati negli ultimi anni. Questo lavoro di tesi si pone quindi l'obbiettivo di esplorare e sviluppare un modello per il riconoscimento di errori
  in un qualsiasi file sorgente scritto in linguaggio sia C sia \CPP.
\end{abstract}
\begingroup\selectlanguage{italian}

    \tableofcontents
    \listoffigures
    \listoftables

    \mainmatter
    \chapter*{Introduzione}

\begin{figure}[h]
    \centering
    \begin{tikzpicture}[block/.style={draw, rectangle, minimum height=1cm}]
        \tikzset{vertex/.style = {shape=circle,draw,minimum size=1.5em}}
        \tikzset{edge/.style = {->,> = latex'}}
      
        % vertices
      \node[block] (code) at  (0,0) {code snippets};
      \node[block] (contexts) at (4,0) {ast contexts};
      \node[block] (vector) at (8,0) {feature vector}; 

      \node[block] (classificazione) at (12,1) {classificazione};
      \node[block] (regressione) at (12, -1) {regressione};

     \draw[edge] (vector) to (classificazione);
     \draw[edge] (vector) to (regressione);



      \draw[edge] (code) to (contexts);

      \draw (contexts) -- (vector) node (arco) [midway] {};

      \node [fit=(code) (contexts),draw,dotted,red, ultra thick,label={[red]Generazione dataset}] {};
      \node [fit=(vector) (arco), draw,dotted,orange, ultra thick,label={[orange]Code2Vec}] {};
      \node [fit=(vector) (classificazione) (regressione) (arco), draw,dotted, blue, ultra thick,label={[blue]Deep learning}] {};

      \end{tikzpicture}
      \caption{Struttura del modello}
      \label{fig:struttura_generale}
    \end{figure}


    \begin{figure}[h]
        \centering
        \scalebox{0.8}{
            \begin{tikzpicture}[block/.style={draw, rectangle, minimum height=1cm}]
                \tikzset{vertex/.style = {shape=circle,draw,minimum size=1.5em}}
                \tikzset{edge/.style = {->,> = latex'}}
    
                \node[vertex] (xs0) at (0,3) {$x_s^0$};
                \node[vertex] (p0) at (0,1) {$p^0$};   
                \node[vertex] (xt0) at (0,-1) {$x_t^0$};
                
                \node at (1,3) {...};
                \node at (1,1) {...};
                \node at (1,-1) {...};
    
    
                \node[vertex] (xsc) at (2,3) {$x_s^c$};
                \node[vertex] (pc) at (2,1) {$p^c$};   
                \node[vertex] (xtc) at (2,-1) {$x_t^c$};
    
    
                \node[vertex] (mask) at (1,-3) {$m$};
    
    
                \node[vertex] (xs) at (4,3) {$x_s$};
                \node[vertex] (p) at (4,1) {$p$};   
                \node[vertex] (xt) at (4,-1) {$x_t$};
    
    
    
                \node[block] (classificazione) at (15, 2) {classificazione};
                \node[block] (regressione) at (15, -2) {regressione};
    
                \node (input) at(3.5,0) {};
    
                \node[rectangle, draw,dotted,red, ultra thick,label={[red]code2vec}, minimum width=6cm,
                minimum height = 2cm] at(8,0) (code2vec) {};
    
                \draw  [decorate,decoration={brace,amplitude=10pt}] (2.75,3.5) -- (2.75,-1.5) node (vec) [black,midway,xshift=0.4cm, yshift=1.8cm, label={[rotate=-90]right:contexts vector}] {};
                
                \node [fit=(xs0) (p0) (xt0) (mask) (xsc) (pc) (xtc),draw,dotted,blue, thick,label={[blue]input}] {};
                %\node [fit=(code2vec) (vector),draw,dotted,red, ultra thick,label={[red]code2vec}] (code2vec) {};
                \node [fit=(classificazione) (regressione),draw,dotted,orange, ultra thick,label={[orange]predizione}] {};
            
                \draw[edge] (code2vec) to (classificazione);
                \draw[edge] (code2vec) to (regressione);
    
                \draw[edge] (input) to (code2vec);
                \end{tikzpicture}
            }
          \caption{Struttura del modello utilizzato}
          \label{fig:struttura}
    \end{figure}

    \chapter{Introduzione teorica}\label{chap:introduzione_teorica}

\section{Code2Vec}\label{sec:code2vec}

\subsection{AstContext}\label{subsec:astcontext}
    \cite{Dirac19}

    \cleardoublepage
    \bibliographystyle{plain}
    \addcontentsline{toc}{chapter}{\bibname}
    \bibliography{main}
\end{document}
