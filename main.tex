% !TeX root = main.tex
%! Author = Matteo
%! Date = 10/06/2022

% Preamble
%\documentclass[11pt]{book}

\documentclass[12pt, letterpaper, openright, a4paper]{report}
% Packages
\usepackage{amsmath}
\usepackage[italian,english]{babel}
\usepackage{newlfont}
\usepackage{gensymb}
\usepackage{lipsum}
\usepackage{hyperref}
\usepackage{xr}
\usepackage{graphicx}
\usepackage{caption,subcaption}
\usepackage{amsfonts}
\usepackage{listings}
\usepackage{mathtools}
\usepackage{footmisc}
%Spacing del testo
\usepackage{setspace} 
\onehalfspacing

%Per tabelle
\usepackage{caption, tabularx, ragged2e, array}

%Per i plot
\usepackage{pgfplots}

%Gestione degli alberi delle strutture delle cartelle
\usepackage{forest}

\definecolor{folderbg}{RGB}{124,166,198}
\definecolor{folderborder}{RGB}{110,144,169}


\def\Size{4pt}
\tikzset{
      folder/.pic={
        \filldraw[draw=folderborder,top color=folderbg!50,bottom color=folderbg]
          (-1.05*\Size,0.2\Size+5pt) rectangle ++(.75*\Size,-0.2\Size-5pt);  
        \filldraw[draw=folderborder,top color=folderbg!50,bottom color=folderbg]
          (-1.15*\Size,-\Size) rectangle (1.15*\Size,\Size);
      }
    }

%Comando per fare reference ad altri file
\makeatletter
\newcommand*{\addFileDependency}[1]{% argument=file name and extension
  \typeout{(#1)}
  \@addtofilelist{#1}
  \IfFileExists{#1}{}{\typeout{No file #1.}}
}
\makeatother

\newcommand*{\myexternaldocument}[1]{%
    \externaldocument{#1}%
    \addFileDependency{#1.tex}%
    \addFileDependency{#1.aux}%
}

%Comando per scrivere c++
\newcommand{\CPP}{C\nolinebreak\hspace{-.05em}\raisebox{.4ex}{\tiny\bf +}\nolinebreak\hspace{-.10em}\raisebox{.4ex}{\tiny\bf +}}
\def\CPP{{C\nolinebreak[4]\hspace{-.05em}\raisebox{.4ex}{\tiny\bf ++}}}

%Comandi per scrivere parole in inglese
\newcommand{\ML}{\textit{Machine Learning}}
\newcommand{\DL}{\textit{Deep Learning}}
\newcommand{\DS}{\textit{Dataset}}
\newcommand{\ds}{\textit{dataset}}

%COmandi per accenti
\newcommand{\Eaccentata}{\'E}

%Comando per mettere le angle bracket e minore e maggiore
\newcommand{\<}{\textless}
%\newcommand{\>}{\textgreater}

\newcommand{\anglebra}[1]{\textlangle{#1}\textrangle}

%Comando per typesettare comandi
\newcommand{\command}[1]{\[\$ \; \textit{#1}\]}
\newcommand{\args}[1]{\textless #1\textgreater}

%Comando per typesettare codice

\newcounter{code}
\renewcommand{\lstlistingname}{Snippet}
\lstdefinestyle{verbo}{  % code typesetting optins
    basicstyle=\small\ttfamily,
    breaklines=true,
    numbers=left,
    breakatwhitespace=true,
    language=Python,
    tabsize=1,
    resetmargins=true,
    xleftmargin=0pt,
    frame=single,
    showstringspaces=false
}
\lstnewenvironment{code}[1][]
 {\lstset{style=verbo,#1}}
 {}

%Comandi per la generazione di grafi
\usepackage{tikz}
\usetikzlibrary{arrows}


%Comandi per i record di dati
\newcommand{\record}[1]{\anglebra{\textit{#1}}}

%Comando per argmax e min
\DeclareMathOperator*{\argmax}{arg\,max}
\DeclareMathOperator*{\argmin}{arg\,min}


%Comando per il referecing di capitoli
\NewDocumentCommand{\chapref}{s m}{Chapter~\ref{#2}\IfBooleanF{#1}{ \nameref{#2}}}

% Per il frontespizio
\textwidth=450pt\oddsidemargin=0pt\evensidemargin=0pt

% Documents
\begin{document}
    \begin{titlepage}
\begin{center}
{{\Large{\textsc{Alma Mater Studiorum $\cdot$ Universit\`a di
Bologna}}}} \rule[0.1cm]{15.8cm}{0.1mm}
\rule[0.5cm]{15.8cm}{0.6mm}
{\small{\bf SCUOLA DI SCIENZE\\
Corso di Laurea Triennale in Informatica }}
\end{center}
\vspace{15mm}
\begin{center}
{\LARGE{\bf Tecniche di deep learning}}\\
\vspace{3mm}
{\LARGE{\bf per il riconoscimento}}\\
\vspace{3mm}
{\LARGE{\bf di errori nei programmi}}\\
\end{center}
\vspace{40mm}
\par
\noindent
\begin{minipage}[t]{0.47\textwidth}
{\large{\bf Relatore:\\
Chiar.mo Prof.\\
Maurizio Gabbrielli}}
\end{minipage}
\hfill
\begin{minipage}[t]{0.47\textwidth}\raggedleft
{\large{\bf Presentata da:\\
Matteo Vannucchi}}
\end{minipage}
\vspace{20mm}
\begin{center}
{\large{\bf Sessione I\\%inserire il numero della sessione in cui ci si laurea
Anno Accademico 2021-2022}}%inserire l'anno accademico a cui si è iscritti
\end{center}
\end{titlepage}

    \begingroup\selectlanguage{english}
\begin{abstract}
  Il ruolo dell'informatica, in un mondo in progressiva digitalizzazione di ogni singolo aspetto della vita dell'individuo, è ormai diventato chiave del suo funzionamento. 
  Con l'aumentare della complessità del codice e delle dimensioni dei progetti, il rilevamento di errori diventa sempre di più un'attività difficile e lunga. Meccanismi di analisi
  del codice sorgente tradizionali sono esistiti fin dalla nascita dell'informatica stessa, e il loro ruolo all'interno della catena produttiva di un team di programmatori non è mai stato cosi fondamentale come 
  lo è tuttora. Questi meccanismi di analisi però non sono esenti da problematiche: il tempo di esecuzione su progetti grandi e la percentuale di falsi positivi possono infatti diventare un grosso problema.
  Per questi motivi meccanismi fondati su \ML, e più in particolare \DL, sono stati sviluppati negli ultimi anni. Questo lavoro di tesi si pone quindi l'obbiettivo di esplorare e sviluppare un modello per il riconoscimento di errori
  in un qualsiasi file sorgente scritto in linguaggio sia C sia \CPP.
\end{abstract}
\begingroup\selectlanguage{italian}

    \tableofcontents
    \listoffigures
    \listoftables

    \chapter*{Introduzione}

\begin{figure}[h]
    \centering
    \begin{tikzpicture}[block/.style={draw, rectangle, minimum height=1cm}]
        \tikzset{vertex/.style = {shape=circle,draw,minimum size=1.5em}}
        \tikzset{edge/.style = {->,> = latex'}}
      
        % vertices
      \node[block] (code) at  (0,0) {code snippets};
      \node[block] (contexts) at (4,0) {ast contexts};
      \node[block] (vector) at (8,0) {feature vector}; 

      \node[block] (classificazione) at (12,1) {classificazione};
      \node[block] (regressione) at (12, -1) {regressione};

     \draw[edge] (vector) to (classificazione);
     \draw[edge] (vector) to (regressione);



      \draw[edge] (code) to (contexts);

      \draw (contexts) -- (vector) node (arco) [midway] {};

      \node [fit=(code) (contexts),draw,dotted,red, ultra thick,label={[red]Generazione dataset}] {};
      \node [fit=(vector) (arco), draw,dotted,orange, ultra thick,label={[orange]Code2Vec}] {};
      \node [fit=(vector) (classificazione) (regressione) (arco), draw,dotted, blue, ultra thick,label={[blue]Deep learning}] {};

      \end{tikzpicture}
      \caption{Struttura del modello}
      \label{fig:struttura_generale}
    \end{figure}


    \begin{figure}[h]
        \centering
        \scalebox{0.8}{
            \begin{tikzpicture}[block/.style={draw, rectangle, minimum height=1cm}]
                \tikzset{vertex/.style = {shape=circle,draw,minimum size=1.5em}}
                \tikzset{edge/.style = {->,> = latex'}}
    
                \node[vertex] (xs0) at (0,3) {$x_s^0$};
                \node[vertex] (p0) at (0,1) {$p^0$};   
                \node[vertex] (xt0) at (0,-1) {$x_t^0$};
                
                \node at (1,3) {...};
                \node at (1,1) {...};
                \node at (1,-1) {...};
    
    
                \node[vertex] (xsc) at (2,3) {$x_s^c$};
                \node[vertex] (pc) at (2,1) {$p^c$};   
                \node[vertex] (xtc) at (2,-1) {$x_t^c$};
    
    
                \node[vertex] (mask) at (1,-3) {$m$};
    
    
                \node[vertex] (xs) at (4,3) {$x_s$};
                \node[vertex] (p) at (4,1) {$p$};   
                \node[vertex] (xt) at (4,-1) {$x_t$};
    
    
    
                \node[block] (classificazione) at (15, 2) {classificazione};
                \node[block] (regressione) at (15, -2) {regressione};
    
                \node (input) at(3.5,0) {};
    
                \node[rectangle, draw,dotted,red, ultra thick,label={[red]code2vec}, minimum width=6cm,
                minimum height = 2cm] at(8,0) (code2vec) {};
    
                \draw  [decorate,decoration={brace,amplitude=10pt}] (2.75,3.5) -- (2.75,-1.5) node (vec) [black,midway,xshift=0.4cm, yshift=1.8cm, label={[rotate=-90]right:contexts vector}] {};
                
                \node [fit=(xs0) (p0) (xt0) (mask) (xsc) (pc) (xtc),draw,dotted,blue, thick,label={[blue]input}] {};
                %\node [fit=(code2vec) (vector),draw,dotted,red, ultra thick,label={[red]code2vec}] (code2vec) {};
                \node [fit=(classificazione) (regressione),draw,dotted,orange, ultra thick,label={[orange]predizione}] {};
            
                \draw[edge] (code2vec) to (classificazione);
                \draw[edge] (code2vec) to (regressione);
    
                \draw[edge] (input) to (code2vec);
                \end{tikzpicture}
            }
          \caption{Struttura del modello utilizzato}
          \label{fig:struttura}
    \end{figure}

    \chapter{Introduzione teorica}\label{chap:introduzione_teorica}

\section{Code2Vec}\label{sec:code2vec}

\subsection{AstContext}\label{subsec:astcontext}
    \chapter{Dataset}\label{chap:dataset}
In questo capitolo tratteremo la generazione del dataset posto alla base del modello che andremo a creare poi nel \autoref{chap:modello}, vederemo prima il dataset originale utilizzato e poi 
come è stato aumentato tramite l'utilizzo di ulteriori analizzatori statici per migliorarne la precisione delle rilevazioni, andando a ridurre il numero di falsi positivi.
Verrà poi presentato come le rilevazioni degli analizzatori statici sono utilizzate per la associazione fra un \textit{code snippet} e il relativo errore, poi come da quest'ultimo venga ricavato il codice in formato di \textit{ast context vector}.

\section{Dataset originale}
Come detto in precedenza questo dataset non è stato generato partendo da zero ma facendo riferimento al dataset creato da \cite{gelman2019source}. Il dataset consiste di circa 3000 progetti di GitHub, scritti in linguaggi C e \CPP,
 che rispettano due requisiti: hanno una licenza ridistribuibile e hanno almeno 10 stelle.
 Il secondo requisito ci serve per garantire che i progetti all'interno del dataset soddisfino dei requisiti di qualità, infatti come precedenti studi hanno mostrato (come ad esempio \cite{papamichail2016user}) si può utilizzare il numero
 di stelle su GitHub come un \textit{proxy} per la qualità del codice stesso.

Il dataset contiene per ogni progetto una serie di analisi effettuate: l'analisi di Doxygen che estrae le coppie codice-commento e l'analisi di Infer che produce un report di analisi statica degli errori.
Visto l'utilizzo che ne sarebbe stato fatto di questo dataset l'analisi di Doxygen è stata scartata. In \autoref{fig:dir_struct} si può vedere la struttura tipica di uno dei circa 3000 progetti presenti.
\begin{figure}
    \centering
    \scalebox{0.6}{
        \begin{forest}
            for tree={
                font=\ttfamily,
                grow'=0,
                child anchor=west,
                parent anchor=south,
                anchor=west,
                calign=first,
                inner xsep=7pt,
                edge path={
                  \noexpand\path [draw, \forestoption{edge}]
                  (!u.south west) +(7.5pt,0) |- (.child anchor) pic {folder} \forestoption{edge label};
                },
                % style for your file node 
                file/.style={edge path={\noexpand\path [draw, \forestoption{edge}]
                  (!u.south west) +(7.5pt,0) |- (.child anchor) \forestoption{edge label};},
                  inner xsep=2pt,font=\small\ttfamily
                             },
                before typesetting nodes={
                  if n=1
                    {insert before={[,phantom]}}
                    {}
                },
                fit=band,
                before computing xy={l=15pt},
              }  
        [Project-name
          [source
            [Project-name
                [Makefile, file]
                [File1.c, file]
                [File2.c, file]
                [..., file]
                [Folder1]
                [Folder2]
                [...]
            ]
          ]
          [derivatives
            [
                Infer-out
                [bugs.txt, file]
            ]
            ]
            [LICENSE, file]
            [url, file]
        ]
        \end{forest}
    }
    %\includegraphics[scale=0.3]{images/immagineStrutturaDirectoryIniziale.png}
    \caption{La struttura della directory di un progetto del dataset iniziale}
    \label{fig:dir_struct}
\end{figure}

Come si può notare ogni progetto contiene anche un Makefile, elemento fondamentale perché gli analizzatori statici che andremo ad aggiungere spesso richiedono l'esistenza di un Makefile funzionante.

%Figura della struttura delle directory


%\subsection{Dataset originale}
 

%\subsection{Analizzatori utilizzati}

\section{Analizzatori di codice statici}
Un'analizzatore di codice è un programma che prende in input uno o più file e genera un report degli errori, cioè una lista di coppie del tipo $<$Errore, Posizione$>$. Di questi analizzatori ne esistono due macro categorie: statici e dinamici. 
Gli analizzatori statici sono programmi che effettuano controlli solo sul codice a livello testuale e che quindi non eseguono in nessuna maniera il codice, gli analizzatori dinamici sono invece analizzatori più complessi che effettuano controlli a \textit{run-time}
andando quindi ad'eseguire il codice stesso.

Gli analizzatori non sono però perfetti, infatti nell'insieme degli errori trovati si possono spesso trovare dei falsi positivi, cioè frammenti di codice segnati come erronei ma che in realtà non presentato nessun tipo di problema. Scopo appunto del dataset aumentato
è quello di ridurre il numero di falsi positivi.


\subsection{Analisi a livello di progetto}
La maggior parte degli analizzatori statici inoltre è in grado di lavorare a livello di progetti, andando quindi a risolvere correttamente gli \textit{include} (nel caso di C e \CPP), e quindi generando un output più significativo. 
Alcuni di questi per far ciò hanno bisogno di quello che viene chiamato \textit{compilation database} e, per soddisfare questo requisito, esistono strumenti appositi che utilizzano il Makefile per generarlo, nel caso di questo lavoro è stato utilizzato un programma
chiamato Bear


\subsection{Analizzatori ulteriori utilizzati}
Come analizzatori statici ulteriori da aggiungere in più a Infer, di cui ogni elemento del dataset ha già l'analisi sua associata, sono stati scelti i seguenti tre:
    \begin{itemize}
        \item L'analizzatore Cppcheck che, a detta degli autori, ha come scopo principale il ridurre il numero di falsi positivi.
        \item Il compilatore GCC che nonostante sia un compilatore vero e proprio ha anche funzionalità per l'analisi statica dei programmi attraverso la flag \textit{-fanalyzer}.
        \item Similarmente a GCC come terzo viene scelto il compilatore Clang che attraverso un suo tool chiamato Clang-Check è in grado di effettuare analisi statiche.
    \end{itemize}
Non sono invece stati usati analizzatori dinamici, questo perché il loro utilizzo in modo automatizzato è un operazione complicata se non quasi impossibile. 
Infatti quasi tutti i programmi prendono o dei parametri all'esecuzione o degli input durante l'esecuzione, ma fornire questi dati in modo consistente e sensato per il programma e in modo automatizzato rende il tutto veramente difficile.

L'utilizzo di essi però potrebbe portare a risultati molto interessanti poiché parte dei falsi positivi degli analizzatori statici deriva dal non poter decidere se frammenti di programmi sono o non sono eseguiti e quindi gli analizzano tutti, ma quello che può succedere
è che se in un frammento di programma che non viene sicuramente mai eseguito c'è un errore, l'analizzatore statico lo riferisce quello dinamico, correttamente, no.

%. La maggior parte degli analizzatori statici infatti


%\subsection{Fase 1: utilizzo di analizzatori statici per la generazione di report di errori}

%\subsection{Fase 2: aggregazione dei report}

%\subsbusection{Conversione degli errori}

%\subsection{Fase 3: }

%\subsection{Fase 4: generazione degli Ast Context}

%\section{Statistiche finali del dataset}
    \chapter{Il modello predittivo}\label{chap:modello}
Nel seguente capitolo affronteremo lo sviluppo del modello predittivo. 
Vedremo, prima di tutto, la struttura del modello discutendone i principali componenti e varie iterazioni di essa.  
In un secondo momento vedremo un problema fondamentale dato dalla distribuzione del dataset: il problema dello sbilanciamento.
Verrà anche introdotto brevemente come viene addestrato e le metriche utilizzato per valutarlo.
Infine saranno discussi i risultati ottenuti. 

\section{Struttura}
Come già introdotto nel \autoref{chap:introduzione_teorica}, questo modello si basa su un meccanismo di codifica del codice separato in due fasi:
    \begin{itemize}
        \item La prima codifica del \textit{code snippet} in un vettore di \textit{ast contexts}, effettuata a tempo di creazione del dataset, come già discusso nel \autoref{chap:dataset}.
        \item La seconda codifica del vettore di \textit{ast contexts} in un vettore di \textit{feature} attraverso meccanismi di \DL.
    \end{itemize}
Una volta ottenuto il vettore delle feature, vengono utilizzati due 'sotto reti' per la classificazione e la regressione. 
Possiamo vedere riassunta a grandi linee la struttura della rete in \autoref{fig:struttura}.

\begin{figure}[h]
    \centering
    \scalebox{0.8}{
        \begin{tikzpicture}[block/.style={draw, rectangle, minimum height=1cm}]
            \tikzset{vertex/.style = {shape=circle,draw,minimum size=1.5em}}
            \tikzset{edge/.style = {->,> = latex'}}
            
            \node[block, label={Input}] (input) at (0,0) {ast contexts vector};
            \node[block] (code2vec) at (4,0) {code2vec};

            \node[block] (classificazione) at (8, 2) {classificazione};
            \node[block] (regressione) at (8, -2) {regressione};
            
            \draw [edge] (input) to (code2vec);
            \draw [edge] (code2vec) to (classificazione);
            \draw [edge] (code2vec) to (regressione);
            
            \end{tikzpicture}
        }
      \caption{Struttura astratta del modello utilizzato}
      \label{fig:struttura}
\end{figure}

Nelle successive sezione discuteremo, in maniera approfondita, le seguenti tematiche:
    \begin{itemize}
        \item La struttura degli input e come sono stati gestiti i cambiamenti della loro forma discussi in precedenza nel \autoref{chap:dataset}.
        \item La struttura del modello di classificazione.
        \item La struttura del modello di regressione.
    \end{itemize}


\subsection{Struttura degli input}
Il dataset generato nel \autoref{chap:dataset} contiene per ogni suo elemento un vettore di \textit{ast contexts}, cioè un vettore di triple della forma:
    \[(x_s^{(i)}, p^{(i)}, x_t^{(i)})\]
tali per cui vale la seguente relazione:
    \[x_s^{(i)}, x_t^{(i)} \in \mathbb{N}^{l}, \quad p^{(i)} \in \mathbb{N}^{k}\]
dove $l$ e $k$ rappresentano rispettivamente la lunghezza massima del vettore dei token di inizio/fine e la lunghezza massima del vettore dei cammini\footnote{Nel caso in cui non siano effettivamente lunghi $l$ o $k$ vengono ridimensionati tramite del \textit{padding}},
fissate al momento della creazione del dataset (vedremo in seguito che valori sono stati assegnati e provati).


Prima però di poter utilizzare questo vettore come input del modello, deve essere trasformato in tre vettori separati della seguente forma:
\[x_s, x_t \in \mathbb{N}^{c \times l}, \quad p \in \mathbb{N}^{c \times k}\]
dove la constante $c$ rappresenta la lunghezza massima dei vettori di \textit{ast contexts} (di nuovo in seguito vedremo i suoi valori).
Definiamo i tre vettori nel seguente modo:
    \begin{align*}
        x_s &= (x_s^{(0)}, x_s^{(1)}, ..., x_s^{(c)}) \\
        x_t &= (x_t^{(0)}, x_t^{(1)}, ..., x_t^{(c)}) \\
        p &= (p^{(0)}, p^{(1)}, ..., p^{(c)}) 
    \end{align*}
Può succedere, però, che un vettore di \textit{ast contexts} abbia una lunghezza $c^{\prime} < c$.
In questo caso dovremo andare ad aggiungere $c - c^{\prime}$ \textit{ast contexts} di \textit{padding} che saranno rappresentati da specifiche triple di vettori di token che, nel rispettivo vocabolario, rappresentano dei token di \textit{padding} (saranno dei token \<PAD\textgreater).

Una volta fatto questo dobbiamo però indicare al modello quali degli \textit{ast contexts} sono di \textit{padding}. 
Per far ciò introduciamo l'ultimo dei quattro input del modello: la maschera.
La maschera sarà un vettore $m$ di lunghezza $c$ definito nel seguente modo:
    \begin{align*}
        m_i =
        \begin{cases*}
        1 & se l'elemento $i$-esimo non è padding \\
        0 & altrimenti
        \end{cases*}
    \end{align*}

\subsection{Gestione cambiamenti della forma}\label{subsec:cambiamenti_forma}
Una volta trasformato l'input avremo quindi tante quadruple della forma:
    \[(x_s, p, x_t, m)\]
tale per cui:
\[x_s, x_t \in \mathbb{N}^{c \times l}, \quad p \in \mathbb{N}^{c \times k}, \quad m \in  \mathbb{N}^{c}\]
All'interno del modello, la prima trasformazione che avviene è quella dell'\textit{embedding} dei tre vettori di token attraverso un \textit{layer} specifico.
Il risultato di ciò sono dei vettori della forma:
    \[x_s^{\prime}, x_t^{\prime} \in \mathbb{N}^{c \times l \times d}, \quad p^{\prime} \in \mathbb{N}^{c \times k \times d}\]
dove $d$ è la dimensione dell'\textit{embedding} (nota: $d$ può essere diverso per $p$, $x_s$ e $x_t$).
Nella studio di code2vec \cite{alon2019code2vec}, come era già stato discusso nel \autoref{chap:dataset}, i vettori d'input hanno una forma leggermente diversa:
\[x_s, x_t \in \mathbb{N}^{c}, \quad p \in \mathbb{N}^{c}, \quad m \in  \mathbb{N}^{c}\]
ottenendo successivamente al \textit{layer} di \textit{embedding}:
\[x_s^{\prime}, x_t^{\prime} \in \mathbb{N}^{c \times d}, \quad p^{\prime} \in \mathbb{N}^{c \times d}\]
Per uniformare quindi i valori a come quelli usati dalla ricerca, effettueremo un appiattimento dei vettori post-\textit{embedding}, ottenendo:
\[x_s^{\prime\prime}, x_t^{\prime\prime} \in \mathbb{N}^{c \times (l \cdot d)}, \quad p^{\prime\prime} \in \mathbb{N}^{c \times (k \cdot d)}\]


\subsection{Classificazione}
L'obbiettivo della classificazione in questo modello è il predire la classe di errore o l'assenza di errore. 
Il modello, di conseguenza, in output dovrà fornire un vettore $c$ tale per cui per ogni $i$:
\[0 \leq c_i \leq 1\]
avremo quindi che il vettore $c$ è una \textit{distribuzione di probabilità} delle classi da predire.
Di conseguenza la classe con maggior probabilità sarà la classe predetta, cioè:
    \[\argmax_i c_i\]
Nel lavoro svolto la rete di classificazione prenderà in input il vettore delle \textit{feature} prodotto dal modello di code2vec.
Questo vettore viene dato in input ad'una serie di \textit{hidden dense layer} culminanti in un \textit{layer} di predizione che utilizza come funzione di attivazione la funzione \textit{softmax}, andando a produrre il vettore $c$.

Visto l'output che produce questo modello, prima di poter computare la funzione di \textit{loss} dovremo trasformare il \textit{label} associato al \textit{code snippet} in una versione \textit{one-hot encoded}. 




\subsection{Regressione}
Il modello della regressione ha come scopo il predire il numero della riga dell'eventuale errore.
La struttura utilizzata è molto semplice: un unico \textit{dense layer} che prende in input il vettore delle \textit{feature} con un singolo output.

Similmente alla classificazione, anche per la regressione dobbiamo processare la riga dell'errore associata al \textit{code snippet}. 
Per evitare di avere una varianza troppo grande, con a volte numeri di riga molto bassi e a volte molto alti, il valore viene normalizzato da un fattore costante tale da rendere ogni singolo valore compreso tra 0 e 1.



\section{Sbilanciamento del dataset}\label{sec:sbilanciamento}
Illustriamo ora il problema principale in cui ci si è imbattuti nel realizzare questo modello: lo sbilanciamento del dataset.
Un dataset, in un problema di classificazione, si definisce sbilanciato se le proporzioni del numero di campioni per ogni classe hanno grosse differenze.
Nel nostro caso possiamo vedere ciò in \autoref{fig:sbilanciamento}. 
Come si può notare la classe dell'assenza di errori rappresenta circa il 95\% del dataset, mentre il restante 5\% è suddiviso fra le 16 classi possibili di errori.


\begin{figure}[h]
    \centering
    \begin{tikzpicture}
        \begin{axis}[
            width  = \textwidth * 0.7,
            height = 7cm,
            major x tick style = transparent,
            ybar=0.1pt,
            bar width=10pt,
            ymajorgrids = true,
            ylabel style={yshift=2ex},
            xlabel=Classe,
            ylabel=Numero di samples,
            xtick = data,
            scaled y ticks = false,
            ymin=0,ymax=42000,
            ytick style={draw=none},
            ]
    \addplot table {
      Classe Numero
      0 41417 
      1 1032
      2 911
      3 792
      4 325
      5 122
      6 85
      7 83
      8 80
      9 56
      10 31
      11 16    
      12 10
      13 4
      14 3
      15 2
      16 1      
    };
    \end{axis}
    \end{tikzpicture}

    \caption{Figura che mostra quanto il dataset sia sbilanciato sia verso la classe dell'assenza di errori sia internamente fra le classi di errori}
    \label{fig:sbilanciamento}
\end{figure}

Il problema dello sbilanciamento è molto grave poiché rende difficile sia l'addestramento della rete sia la sua valutazione, vediamo ora un esempio di ciò.
Supponiamo di creare un modello che predice, per ogni input datogli, sempre l'assenza di errore: col nostro dataset questo modello avrebbe una precisione del 95\%.
Vedendo solo questa metrica potrebbe quindi sembrare essere un modello quasi ideale, mentre invece ovviamente non lo è.
Vedremo nel \autoref{subsec:metriche} come esistono delle metriche in grado di essere utili nonostante lo sbilanciamento.

L'addestramento, invece, è reso difficile dal momento che, con le funzioni di \textit{loss} utilizzate, solitamente il modello tenderà a diventare come quello descritto sopra.
Verranno quindi utilizzate una serie di tecniche nel tentativo di ridurre gli effetti dello sbilanciamento, in particolare vedremo:
    \begin{itemize}
        \item L'utilizzo di una funzione di \textit{loss} pesata.
        \item L'utilizzo di \textit{oversampling}.
        \item La riduzione del numero di classi da predire. 
    \end{itemize}
Nell'addestramento del modello finale verranno utilizzate sia la seconda sia la terza tecnica.

\subsection{Loss pesata}
Il meccanismo di \textit{loss} pesata funziona in modo molto semplice: assegnare ad'ogni classe peso diverso nella computazione della \textit{loss}.
Facendo così, se si sono assegnati i pesi corretti, si avrà che le classi minoritarie avranno molto più peso rispetto a quelle maggioritarie e quindi, nel caso in cui il modello sbagli a predire una delle classi minoritarie, la perdita sarà maggiore.

In questo lavoro è stata implementata attraverso l'utilizzo di una \textit{matrice dei pesi} $W$ della forma:
    \[W \in \mathbb{R}^{n \times n}\]
dove $n$ rappresenta il numero di classi.
La semantica di questa matrice $W$ è la seguente: il valore $W_{i,j}$ indica il peso di un \textit{sample} di classe $i$ classificato erroneamente come di classe $j$. 
Definendo $f_i$ come la frequenza assoluta della classe $i$-esima, dovremo avere quindi che:
    \[W_{i,j} \propto \frac{f_j}{f_i}\]
Esistono diverse tecniche per assegnare questi pesi, in questo caso è stata utilizzata la seguente formula:
    \[W_{i,j} = \frac{f_j + \epsilon}{f_i + \epsilon}\]
L'aggiunta di un piccolo valore $\epsilon$ è dovuta al fatto che, in rari casi, $f_i = 0$.

\'E stato deciso di non utilizzare questo sistema poiché i risultati ottenuti non sono stati ottimali, infatti il modello, nei test effettuati, imparava in ogni caso a predire sempre la classe di assenza d'errore.
Una possibile spiegazione di ciò è che nonostante le classi minoritarie avessero un grosso peso, la probabilità di trovarle in un singolo \textit{batch} di addestramento era bassa, ciò implicava uno strano comportamento della funzione di \textit{loss}. 
 
\subsection{Oversampling}
L'\textit{oversampling} è il processo di aumentare artificialmente il numero di osservazioni delle classi minoritarie in modo tale da pareggiarle con quelle maggioritarie.
L'implementazione dell'\textit{oversampling} può avvenire in svariati modi:
    \begin{itemize}
        \item Ripetizione semplice delle osservazioni delle singole classi minoritarie.
        \item Creazione di osservazioni completamente nuove tramite metodi complessi. Un esempio di questo approccio è il metodo denominato \textit{smote}, descritto nella ricerca \cite{chawla2002smote}, che è fra i più popolari.
         Consiste nel creare dati nei segmenti che congiungono i $k$ vicini della medesima classe più vicini nel \textit{feature space}.
    \end{itemize}
In generale utilizzando questa tecnica avremo che l'addestramento del modello diventa più difficile poiché è più probabile che finisca in \textit{overfitting}.
Creando però dati completamente nuovi, ma teoricamente sensati, cioè come fa \textit{smote}, la probabilità di fare \textit{overfitting} è minore.

Nel caso di questo progetto non è stato possibile utilizzare \textit{smote} poiché l'implementazioni nelle librerie più comuni non funzionavano per la struttura dati usata.
Viene, invece, utilizzato il seguente metodo: ripetizione dei dati mischiando però l'ordine degli \textit{ast contexts} all'interno del vettore.
I due vantaggi ottenibile teoricamente tramite questa tecnica sono:
    \begin{itemize}
        \item Diminuire l'\textit{overfitting} riducendo il numero di dati uguali.
        \item L'eliminazione della semantica associata all'ordine degli \textit{ast contexts} all'interno di un vettore. Infatti, per come vengono estratti, l'ordine non ha alcuna importanza.
            Il riordinamento quindi potrebbe fare 'capire' ciò al modello.
    \end{itemize}

\subsection{Riduzione numero di classi}
L'ultima tecnica che andiamo ad'esporre è la riduzione del numero di classi da predire.
Questa tecnica trasforma il problema di classificazione a $n$ classi in un problema di classificazione a $k<n$ classi.
Una volta fissato un $k < n$ verranno determinate le $k - 1$ classi più frequenti (cioè con il numero di osservazioni più alto) e verranno scelte come le nuove classi.
Le restanti classi vengono aggregate in un'unica classe indicante un errore sconosciuto. 
Avremo, quindi, al variare di $k$ i seguenti casi:
    \begin{itemize}
        \item Ponendo $k=2$ avremo un problema di classificazione binaria, in cui viene classificato la presenza o assenza di errore.
        \item Ponendo $k=5$, come utilizzato in questo progetto, avremo la classificazione degli errori più comuni (\textit{memory leak}, \textit{null dereference}, \textit{dead store}), mentre il restante viene classificato come errore sconosciuto.
        \item Ponendo $k$ più vicino al valore di $n$ non avremo grossi cambiamenti.
    \end{itemize}
Possiamo vedere in \autoref{fig:riduzione} come al variare di $k$ cambia la distribuzione delle classi. 

Utilizzando questa tecnica insieme all'\textit{oversampling} si riduce significativamente l'\textit{overfitting}, poiché lo sbilanciamento del dataset è minore.

\begin{figure}[h]
    \centering
    \begin{subfigure}{.5\textwidth}
        \centering
        \begin{tikzpicture}
            \begin{axis}[
                width  = \textwidth * 0.8,
                height = 7cm,
                major x tick style = transparent,
                ybar=0.1pt,
                bar width=10pt,
                ymajorgrids = true,
                ylabel style={yshift=2ex},
                xlabel=Classe,
                ylabel=Numero di samples,
                xtick = data,
                scaled y ticks = false,
                ymin=0,ymax=42000,
                ytick style={draw=none},
                ]
                \addplot table {
                    Classe Numero
                    0 41417 
                    1 3553
                  };
        \end{axis}
        \end{tikzpicture}
    \caption{Riduzione del numero di classi a $k=2$}
    \end{subfigure}%
    \begin{subfigure}{.5\textwidth}
        \centering
        \begin{tikzpicture}
        \begin{axis}[
            width  = \textwidth * 0.8,
            height = 7cm,
            major x tick style = transparent,
            ybar=0.1pt,
            bar width=10pt,
            ymajorgrids = true,
            ylabel style={yshift=2ex},
            xlabel=Classe,
            ylabel=Numero di samples,
            xtick = data,
            scaled y ticks = false,
            ymin=0,ymax=42000,
            ytick style={draw=none},
            ]
            \addplot table {
                Classe Numero
                0 41417 
                1 1032
                2 911
                3 792
                4 818
              };
    \end{axis}
    \end{tikzpicture}

    \caption{Riduzione del numero di classi a $k=5$}
    \end{subfigure}


    \caption{Comparazione tra riduzione a $k=5$ e $k=2$ classi}
    \label{fig:riduzione}
\end{figure}
\pagebreak

\section{Addestramento}
L'addestramento della rete è stato eseguito numerose volte provando valori per gli \textit{iper parametri} ogni volta diversi. 
Questi parametri, nel nostro caso, consistono in:
    \begin{itemize}
        \item Gli iper parametri standard come \textit{learning rate, batch size, epochs, steps per epoch} e \textit{dropout rate} (per il \textit{layers} di \textit{dropout} che discuteremo in \autoref{subsec:overfitting}).
        \item La dimensione degli \textit{embedding} dei token dei cammini e d'inizio/fine, cioè il valore $d$ introdotto in \autoref{subsec:cambiamenti_forma}.
        \item La dimensione del vettore delle \textit{feature} prodotto dal modello code2vec.
        \item Il numero $k$ di classi da utilizzare. 
    \end{itemize}
L'addestramento inoltre è stato eseguito utilizzando come ottimizzatore l'algoritmo Adam. 
Come \textit{loss function} ne vengono utilizzate due, una per la regressione e una per la classificazione, e sono le seguenti:
    \begin{itemize}
        \item La \textit{categorical crossentropy loss} per la classificazione, nel caso si utilizzi un $k>2$, mentre se $k=2$ si utilizza la \textit{binary crossentropy loss}.
        \item Per la regressione viene utilizzata la \textit{mean squared error}.
    \end{itemize}
Il dataset prima di essere utilizzato viene inoltre separato in tre insiemi distinti:
    \begin{itemize}
        \item Il \textit{train set} che è l'effettivo dataset con cui viene addestrata la rete. Questo rappresenta circa l'80\% del dataset intero.
        \item Il \textit{validation set} con cui, ad'ogni epoca, si va a valutare l'addestramento della rete. \Eaccentata{} fondamentale utilizzarlo poiché le metriche che si ottengo sul \textit{train set} non sono affidabili
                poiché la rete potrebbe (e farà) \textit{overfitting}. Questo rappresenta circa il 10\% del dataset.
        \item Il \textit{test set} con cui, a fine addestramento, si valuta di nuovo la rete. Come il \textit{validation set} rappresenta circa il 10\% del dataset.
    \end{itemize}
Requisito fondamentale è che questi tre insiemi devono essere distinti, e cioè non avere elementi in comune. 
Se avessero elementi in comune le metriche ottenute non sarebbero più affidabili. 

\subsection{Overfitting}\label{subsec:overfitting}
Un modello si dice che è in \textit{overfitting} quando si adatta troppo ai dati osservati e quindi perde capacità di generalizzazione.
In particolare nell'ambito del \ML{} e \DL{} succede quando il modello ha risultati molto buoni sul \textit{training set}, mentre significativamente peggiori sul \textit{validation} e \textit{test set}.

Possiamo vedere un esempio creando un modello che cerca di approssimare una funzione (cioè un problema di regressione). 
In \autoref{fig:esempi_overfitting_underfitting} possiamo vedere sia un modello in \textit{underfit} (cioè il contrario di \textit{overfit}), sia un modello ideale che ha approssimato correttamente la funzione e sia il modello in \textit{overfit}.
Se utilizzassimo quest'ultimo modello per predire nuovi valori, ad esempio determinando il valore di $f(x)$ dove $x$ non fa parte dei punti, non sarebbe in grado di generarne di corretti, mentre il modello ideale si.

\begin{figure}[h]
    \centering
    \includegraphics[scale=0.5]{images/esempio_overfitting_resize.png}
    \caption{Esempio di modello in stato di overfitting, underfitting e ottimale}
    \label{fig:esempi_overfitting_underfitting}
\end{figure}
Questo stato può essere causato da una serie di fattori:
    \begin{itemize}
        \item La complessità del modello, cioè il numero di parametri interni, è troppa alta rispetto al numero di osservazioni nel dataset.
        \item La fase di addestramento è stata eseguita per troppo tempo.
        \item Il dataset utilizzato per l'addestramento è troppo piccolo.
    \end{itemize}
Si può rilevare l'\textit{overfitting} guardando l'evoluzione della funzione di costo attraverso l'epoche di addestramento: nel caso in cui le \textit{loss} calcolate sul \textit{validation} e \textit{train set} divergono allora è molto probabile essere in \textit{overfitting}.
Vediamo un esempio di grafico che ci mostra la divergenza fra le due in \autoref{fig:overfitting_grafo_divergenza}

    \begin{figure}[h]
        \centering
        \includegraphics[scale=0.55]{images/LossOverfitting.png}
        \caption{Divergenza fra loss nel validation e train set}
        \label{fig:overfitting_grafo_divergenza}
    \end{figure}
Il modello utilizzato per questo lavoro soffre altamente di \textit{overfitting} e le cause sono principalmente la dimensione del dataset e la complessità del modello.
La prima causa è difficilmente eliminabile visto che andrebbe generato un nuovo dataset, mentre la seconda causa invece verrà affrontata riducendo valori come la dimensione degli \textit{embedding}, del vettore delle \textit{features} e degli input.
In particolare l'ultimo verrà fatto attraverso la riduzione dei valori di $c$, $l$ e $k$ introdotti in \autoref{subsec:cambiamenti_forma}.


\subsection{Metriche utilizzate}\label{subsec:metriche}
Le metriche sono un meccanismo fondamentale nella valutazione di un modello. 
Per questo progetto vengono utilizzate principalmente metriche per la classificazione, mentre per la regressione viene utilizzato solo il valore della funzione di costo.
In particolare vengono utilizzate le seguenti metriche:
    \begin{itemize}
        \item La precisione per ogni singola classe $i$ calcolata nel seguente modo:
            \[P_i = \frac{TP_i}{TP_i + FP_i}\]
            dove $TP_i$ rappresenta il numero di \textit{true positive} per la classe $i$ e $FP$ il numero di \textit{false positive}.
            La precisione identifica quanti elementi classificati come di classe $i$ sono effettivamente di classe $i$.
        \item Il recall per ogni singola classe $i$ calcolato come:
            \[R_i = \frac{TP_i}{TP_i + FN_i}\]
            dove $FN_i$ rappresenta il numero di \textit{false negative} per la classe $i$.+
            Il recall rappresenta quanti degli elementi della classe $i$ sono stati effettivamente individuati.
        \item L'f1-score per ogni classe $i$ che si calcola nel seguente modo:
            \[F1_i = \frac{2 \times P_i \times R_i}{P_i + R_i}\]
            e rappresenta la media armonica fra la precisione e il recall. Viene quindi utilizzata per bilanciare recall e precisione.
        \item L'f1-score globale calcolato come media degli f1-score delle singole classi.
    \end{itemize}
Spesso succede che ci sia un \textit{trade off} tra la precisione e il recall.
In base al tipo di problema che si sta cercando di risolvere verrà prediletta una delle due metriche.
In questo lavoro è stato deciso di prediligere il recall, poiché è, a nostro avviso, più importante essere in grado di rilevare tutti i possibili errori  

\section{Risultati}
\subsection{Risultati dati dal test dataset}
\subsection{Risultati dati su codice creato al momento}



\section{Ulteriore architettura per la classificazione provate}
Come vedremo in seguito, il modello riesce a determinare con sufficiente correttezza se un \textit{code snippet} presenta o no un errore e riesce a catalogare bene il tipo di errore.
Se però deve fare queste predizioni tutte insieme (e cioè deve predire o la classe indicante l'assenza di errore o la classe dell'errore), vedremo, il modello non generalizzerà altrettanto bene.

Per provare a migliorare i risultati del modello, è stato provato a dividere il meccanismo di predizioni in due fasi:
    \begin{itemize}
        \item Determinare la presenza o l'assenza di un errore.
        \item Determinare la classe dell'errore.
    \end{itemize}
In questa modalità qui, quindi, il modello oltre ad'effettuare le regressione esegue due classificazioni: una binaria e una a più classi.

I risultati prodotti da questa versione, però, non sono così distanti dal modello effettivamente usato.
Questo potrebbe essere determinato da un fattore principale: la difficoltà nell'addestramento. 
Infatti, indipendentemente dalla predizione binaria che fa, il modello restituirà sempre in output anche una predizione sulla classe dell'errore e di conseguenza verrà computata la funzione di perdita.
Per questa ragione anche ai frammenti di codice senza errori bisogna associare un vettore per la predizione delle classi, ma dati i problemi dovuti allo sbilanciamento del dataset discussi in \autoref{sec:sbilanciamento},
il modello imparava a predire solo questo vettore (che è sicuramente il più prevalente).

Una possibile miglioria sarebbe di separare completamente i due modelli: uno predice la presenza o no di errori e l'altro predice solamente l'errore e il numero della riga.
Il funzionamento sarebbe poi il seguente: si utilizza il primo modello per primo poi, nel caso di rilevamento di errori, si utilizza il secondo modello per determinarne il tipo.
Non è stata esplorata questa possibilità poiché al di fuori della portata di questo lavoro. 

\section{Utilizzo}
    \chapter{Conclusioni}
In questo elaborato è stata discussa la creazione di un modello per il rilevamento di errori nel codice di programmi scritti in linguaggio C e \CPP.
La necessità di sviluppare un modello di questo genere, quando nel campo sono ormai diffusi analizzatori sia statici sia dinamici, deriva dal tentativo di ridurre il numero di falsi positivi rilevati da essi.
Il lavoro è stato svolto seguendo la falsariga del operato svolto in \cite{alon2019code2vec}, nella quale viene sviluppato un modello, sempre in \DL{}, in grado di processare codice in un formato di vettore di \textit{ast contexts}.
In particolare, questo progetto si è sviluppato in due fasi distinte: la generazione del dataset e la creazione del modello.

Per la generazione del dataset è stata utilizzata come base una collezione di progetti, risultato del lavoro descritto nella ricerca \cite{gelman2019source}.
Per ogni progetto sono stati poi generati dei report di analisi statica che sono stati poi utilizzati per estrarre le coppie \record{code snippet, errore}.
I \textit{code snippet} sono poi stati processati al fine di creare dei vettori di \textit{ast contexts}.

La seconda parte del progetto consisteva nello sviluppo del modello predittivo, il quale obbiettivo è duplice: predire la classe di errore e predire il numero della riga dell'eventuale errore.
Abbiamo poi visto il problema principale riscontrato nell'addestramento, lo sbilanciamento del dataset, ed una serie di metodi per combatterlo. 
In seguito è stato anche esplorato il problema del \textit{overfitting}.
Come però abbiamo visto nella \autoref{sec:risultati}, i risultati ottenuti non sono entusiasmanti. 
Ciò deriva da una combinazione di più fattori:
    \begin{itemize}
        \item La dimensione ridotta del dataset.
        \item La difficoltà nell'addestramento del modello che spesso finisce in stati di \textit{overfitting}.
        \item La complessità intrinseca del problema. Infatti anche per un programmatore esperto potrebbe essere difficile identificare correttamente certi errori.
    \end{itemize}

\section{Miglioramenti possibili e sviluppi futuri}
Dai risultati ottenuti possiamo concludere che il modello, come sviluppato in questo progetto, non è sufficientemente capace di predire gli errori.
Idealmente si vorrebbe sostituire, o almeno affiancare, un modello di questo tipo ai tradizionali analizzatori, ma nelle condizioni attuali non sarebbe possibile.

Come però abbiamo visto nell'introduzione, diversi modelli sia più tradizionali sia di \ML{} e \DL{} sono stati sviluppati con successo. 
Quindi ci sono sicuramente porzioni migliorabili in questo lavoro:
    \begin{itemize}
        \item La dimensione del dataset che, nonostante non sia di piccole dimensioni, potrebbe non essere adatto per una generalizzazione corretta del modello.
        \item La struttura del modello stesso. Infatti, come abbiamo visto, il modello è difficilmente addestrabile ed eccessivamente complesso.
    \end{itemize}
Soluzione alternativa per la struttura del modello potrebbe essere quella descritta in \cite{bui2021infercode}, nella quale viene generato una rappresentazione del codice attraverso l'uso di un \textit{Tree-Based Convolutional Neural Network} già addestrato. 
Il \textit{training} in quel caso potrebbe rivelarsi più semplice, poiché solamente la classificazione deve essere imparata, mentre non la porzione di \textit{encoding} del codice. 

    \cleardoublepage
    \bibliographystyle{plain}
    \addcontentsline{toc}{chapter}{\bibname}
    \bibliography{main}
\end{document}
