\begingroup\selectlanguage{english}
\begin{abstract}
  Il ruolo dell'informatica è ormai diventato chiave del funzionamento del mondo moderno in progressiva digitalizzazione di ogni singolo aspetto della vita dell'individuo.
  %Il ruolo dell'informatica, in un mondo in progressiva digitalizzazione di ogni singolo aspetto della vita dell'individuo, è ormai diventato chiave del suo funzionamento. 
  Con l'aumentare della complessità del codice e delle dimensioni dei progetti, il rilevamento di errori diventa sempre di più un'attività difficile e lunga. Meccanismi di analisi
  del codice sorgente tradizionali sono esistiti fin dalla nascita dell'informatica stessa, e il loro ruolo all'interno della catena produttiva di un team di programmatori non è mai stato cosi fondamentale come 
  lo è tuttora. Questi meccanismi di analisi però non sono esenti da problematiche: il tempo di esecuzione su progetti grandi e la percentuale di falsi positivi possono infatti diventare un grosso problema.
  Per questi motivi meccanismi fondati su \ML, e più in particolare \DL, sono stati sviluppati negli ultimi anni. Questo lavoro di tesi si pone quindi l'obbiettivo di esplorare e sviluppare un modello per il riconoscimento di errori
  in un qualsiasi file sorgente scritto in linguaggio sia C sia \CPP.
\end{abstract}
\begingroup\selectlanguage{italian}